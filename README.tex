% !TeX program = pdfLaTeX
\documentclass[12pt]{article}
\usepackage{amsmath}
\usepackage{graphicx,psfrag,epsf}
\usepackage{enumerate}
\usepackage{natbib}
\usepackage{listings}
\usepackage{textcomp}
\usepackage[hyphens]{url} % not crucial - just used below for the URL
\usepackage{hyperref}

\providecommand{\tightlist}{%
  \setlength{\itemsep}{0pt}\setlength{\parskip}{0pt}}

%\pdfminorversion=4
% NOTE: To produce blinded version, replace "0" with "1" below.
\newcommand{\blind}{0}

% DON'T change margins - should be 1 inch all around.
\addtolength{\oddsidemargin}{-.5in}%
\addtolength{\evensidemargin}{-.5in}%
\addtolength{\textwidth}{1in}%
\addtolength{\textheight}{1.3in}%
\addtolength{\topmargin}{-.8in}%

\begin{document}

\def\spacingset#1{\renewcommand{\baselinestretch}%
{#1}\small\normalsize} \spacingset{1}


%%%%%%%%%%%%%%%%%%%%%%%%%%%%%%%%%%%%%%%%%%%%%%%%%%%%%%%%%%%%%%%%%%%%%%%%%%%%%%

\if0\blind
{
  \title{\bf Excuse me, do you have a moment to talk about version control?}

  \author{
        Jennifer Bryan \thanks{The author gratefully acknowledges the constructive feedback from
reviewers Nicholas Horton and Colin Rundel.} \\
    RStudio and the Department of Statistics, University of British Columbia\\
      }
  \maketitle
} \fi

\if1\blind
{
  \bigskip
  \bigskip
  \bigskip
  \begin{center}
    {\LARGE\bf Excuse me, do you have a moment to talk about version control?}
  \end{center}
  \medskip
} \fi

\bigskip
\begin{abstract}
Abstract abstract abstract.
\end{abstract}

\noindent%
{\it Keywords:} version control
\vfill

\newpage
\spacingset{1.45} % DON'T change the spacing!

\subsection{Why Git?}\label{why-git}

Why would a statistician use version control, such as Git? And what is
the point of hosting your work online, e.g., on GitHub? Could the gains
possibly justify the inevitable pain?

I say yes, with the zeal of the converted.

There are many benefits of using hosted version control in your
statistical practice:

\begin{itemize}
\tightlist
\item
  Doing your work becomes tightly integrated with organizing, recording,
  and disseminating it. It's not a separate, burdensome task you are
  tempted to neglect.
\item
  Collaboration is much more structured, with powerful tools for
  asynchronous work and managing versions.
\item
  The marginal effort required to create a web presence for a project is
  negligible.
\item
  GitHub makes a fantastic course management system for courses that use
  R. You and your students can exchange actual working code and explore
  the associated results.
\item
  By using common mechanics across work modes (research, teaching,
  analysis), you achieve basic competence quickly and avoid the
  demoralizing forget-relearn cycle.
\end{itemize}

\subsection{What is Git?}\label{what-is-git}

\href{http://git-scm.com}{Git} is a \textbf{version control system}. Its
original purpose was to help groups of developers work collaboratively
on big software projects. Git manages the evolution of a set of files --
called a \textbf{repository} or \textbf{repo} -- in a sane, highly
structured way. It is a bit like the ``Track Changes'' feature from
Microsoft Word, but more rigorous, powerful, and scaled up to multiple
files.

Git has been re-purposed by the data science community
\citep{Ram2013, git-for-humans}. In addition to using it for source
code, we use it to manage the motley collection of files that make up
typical data analytical projects, which often consist of data, figures,
reports, and, yes, source code. Even those who identify more as
statistician than data scientist generally have a similar mix of files
that are the artifacts of a project.

A lone ranger, working on a single computer, can benefit from adopting
version control. But not nearly enough to justify the pain of
installation and workflow upheaval. There are much easier ways to get
versioned back ups of files, if that's all you're worried about.

In my opinion, \textbf{for new users}, the pros of Git only outweigh the
cons when you factor in the overhead of communicating and collaborating
with other people, including your future self. And who among us does not
need to do that? Your life is much easier if this is baked into your
workflow, as opposed to being a separate process that you dread and
avoid. Communication and collaboration are the killer apps of version
control. Git enforces a rigorous model of file management, but it is
critical for the distribution of files across different people,
computers, and time.

This has an implication for selecting your first Git projects: in order
to see substantial gain for your pain, you need to pick projects that
require you to share rapidly evolving files with others. It is tempting
to pick a quiet little project that you work on privately, but you risk
missing the main point of formal version control.

Many people who don't use Git unwittingly re-invent a poor man's version
of it. They take an important file and distribute it via email. Various
parties make changes, decorating the file name with initials, dates, and
other descriptors. Before you know it, the original file is the root of
a complicated phylogeny that no amount of ``Track changes'' and good
intentions can resolve.

\emph{If you're looking at the PDF draft, you're missing out on some
figures around here in the GitHub Markdown version that hint at the
figure I plan to embed.}

The Git way is to track the evolution of that file, through a series of
commits, each equipped with an explanatory message and a nickname. All
collaborators sync regularly to a common version, acknowledging that the
difficulty of merging goes up faster than the size of the difference
\citep{good-enough}. Especially important versions get a human-readable
tag, to signal a meaningful milestone. Yes, there is some pain in
adopting the formalism of Git, but it is worth it.

\emph{make a figure related to the above (reference those file names?)
but juxtaposing different workflows, inspiration from the Bartlett Git
for Designers talk}

\includegraphics[width=1\linewidth]{bartlett-copies-vs-commits}

\subsection{Who should read this and what to
expect}\label{who-should-read-this-and-what-to-expect}

The target reader is anyone who does statistical research, analysis, or
instruction. Those whose work is some combination of these three may
find the work style described here especially rewarding.

This article does not provide step-by-step instructions on how to use
Git and GitHub. This format would not be effective, but we do provide
annotated links to such resources in ?the appendix or online
supplement?. Instead, I'll convey what the workflow feels like and what
the payoffs are, with special attention to the statistics and R context.
The goal is to help the Git-curious generate the activation energy
needed to get started.

\subsection{What is GitHub?}\label{what-is-github}

\href{https://github.com}{GitHub} is currently the most popular Git
hosting service. Others include \href{https://about.gitlab.com}{GitLab}
and \href{https://bitbucket.org}{Bitbucket}. These services provide a
home for Git-based projects on the internet. It is a bit like DropBox or
Google Drive, but more structured, powerful, and programmatic.

The remote host acts as the clearinghouse for a Git-managed project.
This allows other people to see the project files, sync up with the
current version, and perhaps even make or propose changes. These hosting
providers offer well-designed web interfaces that are a dramatic
improvement over traditional Unix Git servers. Many operations can be
done entirely in the browser, such as viewing current or past versions
of a file, commenting on a recent change, and editing or adding files.
These hosts also offer granular control over who can see, edit, and
administer a project.

\emph{possible images include a modified version of this atlassian image
(no arrows between people, yes arrow from people to github), somehow
connected to a screenshot of a GitHub repo}

\includegraphics[width=1\linewidth]{atlassian-git-tutorials-syncing}

Even for private solo projects, there are two advantages to keeping a
synced copy on GitHub:

\begin{enumerate}
\def\labelenumi{\arabic{enumi}.}
\item
  When you are new with Git (or, frankly, even when you're not), it's
  common to goof up the Git infrastructure for a project. Note that your
  files can be intact and safe, even while the Git tracking is a bit
  confused. Of course there are official Git remedies, but sometimes the
  easiest fix is to clone a fresh copy from GitHub, patch things up with
  the changes that only exist locally, and move on with your life. This
  workaround obviously requires the existence of a recent copy on
  GitHub.
\item
  The highly functional web interfaces mentioned above are often the
  most pleasant and natural way to navigate and search your files, even
  though all the same information exists locally. It is a pleasure to
  browse through your own work, across multiple projects or files and
  across time, as if it's a well-designed website. You must push your
  work to GitHub to enjoy this.
\end{enumerate}

\textbf{GitHub issues} are another powerful feature of the platform.
Recall that we are repurposing Git, a tool designed to facilitate
software development. The issues for a project are its bug tracker. For
projects that are not pure software development, we co-opt this
machinery to organize our to do list more generally. The basic unit is
an issue and you can interact with one in two ways.

First, issues are integrated into the project's web interface on GitHub,
with a rich set of options for linking to project files and incremental
changes. Second, issues and their associated comment threads appear in
your email, just like regular messages (this can, of course, be
configured). The result is that all correspondence about a project comes
through your normal channels, but is also tracked inside the project
itself, with excellent navigability and search capabilities. For
software, issues are used to track bugs and feature requests. In a data
analysis project, you might open an issue to flesh out a specific
sub-analysis or to develop a complicated figure. In a course, we use
them to manage homework submission, marking, and peer review.

Issues can be assigned to specific people and they can be labelled, e.g.
``bug'', ``simulation-study'', or ``final-exam''. Coupled with the
ability to cross-link issues and the project files or file changes, you
have extraordinary power to document why things have happened in the
past and to organize what needs to happen in the future.

\subsection{Initial system setup}\label{initial-system-setup}

Here is the initial setup process, which you do once or, for some steps,
once per computer.

\begin{itemize}
\item
  Register for a free account with GitHub.
\item
  Install Git. Depending on your OS, Git might already be installed. But
  many of us may need to install it or will choose to update to a more
  recent version. Some basic configuration is critical, such as setting
  your username and email.
\item
  Install a local Git client \emph{(optional but highly recommended}). A
  Git client is software that provides a graphical user interface for
  Git, which is otherwise command-line only. If you are an R user, you
  will find that
  \href{https://www.rstudio.com/products/rstudio-desktop/}{RStudio}
  provides a great deal of this functionality. There are some notable
  gaps, however, so you might still choose to install a dedicated and
  comprehensive Git client such as
  \href{https://www.sourcetreeapp.com}{SourceTree} or
  \href{https://www.gitkraken.com}{GitKraken}. Git is a file-based
  system, so you can do some operations from RStudio, others from
  SourceTree, and others from the shell.
\item
  Confirm, with a practice repository, that local Git can send and
  receive the current version of the repository on GitHub, known as
  \textbf{pushing} and \textbf{pulling}, respectively. This will require
  authenticating yourself with GitHub from a local shell or Git client.
  At this point, most people elect to do a bit of extra setup to ensure
  that they are not repeatedly challenged for their GitHub credentials
  going forward.
\end{itemize}

Once this setup is done, you are ready to start using Git and GitHub
with your projects.

\subsection{Is this going to hurt?}\label{is-this-going-to-hurt}

Yes.

Git was built neither for the exact usage described here, nor for broad
usability. You will undoubtedly notice this, so it can be helpful to
know this in advance. Happily there are many helpful tools that mediate
your interactions with Git. GitHub itself is a fine example, as is
RStudio. In addition to pointing out tools that soften Git's sharpest
edges, I recommend specific habits and attitudes that reduce
frustration.

General recommendations for agony reduction:

\begin{itemize}
\tightlist
\item
  I repeat: consider using a graphical front-end for Git, a.k.a. a Git
  client, versus restricting yourself to the command line interface.
\item
  Establish confidence in the basics (e.g.~make a change, commit it,
  push it) before wading into more advanced usage (e.g.~branching).
\item
  Commit yourself to Git usage on a project that will provide sustained
  practice over several months. Usage in a course is great, because it
  provides a relentless stream of small deadlines.
\item
  Realize that no one is giving out Git style points. It's ok to
  ``power-cycle'', i.e.~re-initialize the Git repository, to get
  unstuck.
\end{itemize}

\subsection{Repositories and workflow}\label{repositories-and-workflow}

For new or existing projects, you will:

\begin{itemize}
\tightlist
\item
  Dedicate a local directory or folder to it.
\item
  Make it an RStudio Project. \emph{Optional but recommended; obviously
  only applies to projects involving R and users of RStudio.}
\item
  Make it a Git repository.
\end{itemize}

This setup happens once per project and can happen at project inception
or at any later point. Chances are your project already lives in a
dedicated directory. Making this directory an RStudio Project and Git
repository boils down to allowing those applications to leave notes for
themselves in hidden files or directories. The project is still a
regular directory on your computer, that you can locate, name, move, and
generally interact with as you wish. You don't have to handle it with
special gloves!

Here is the daily workflow:

\begin{itemize}
\tightlist
\item
  Go about your usual business, writing R scripts or authoring reports
  in LaTeX or R Markdown. But instead of only \emph{saving} individual
  files, periodically you make a \textbf{commit}, which takes a snapshot
  of all the files in the entire project.

  \begin{itemize}
  \tightlist
  \item
    If you have ever versioned a file
    \href{http://www.phdcomics.com/comics/archive.php?comicid=1531}{by
    adding your initials or the date}, you have effectively made a
    commit, albeit only for a single file. It is a version that is
    significant to you and that you might want to inspect or revert to
    later.
  \end{itemize}
\item
  Push commits to GitHub periodically.

  \begin{itemize}
  \tightlist
  \item
    This is like sharing a document with colleagues on DropBox or
    sending it out as an email attachment. By pushing to GitHub, you
    make your work and all your accumulated progress accessible to
    others.
  \end{itemize}
\end{itemize}

This is a moderate change to your normal, daily workflow. It feels weird
at first, but quickly becomes second nature. In
\href{http://stat545.com}{STAT 545} students are required to submit all
coursework via GitHub, starting in week one. Most have never seen Git
before and do not identify as programmers. It is a major topic in class
and office hours for the first two weeks. Then we practically never
discuss it again.

\subsection{Commits, diffs, and tags}\label{commits-diffs-and-tags}

We now explore the fundamental concepts of Git and connect them to the
data science workflow:

\begin{itemize}
\tightlist
\item
  repository
\item
  commit
\item
  diff
\end{itemize}

Recall that a repository or repo is just a directory of files that Git
will manage holistically. A commit functions like a snapshot of all the
files in the repo, at a specific moment. Under the hood, that is not
exactly how Git implements things. Mental models don't have to be
accurate in order to be useful, but in this case there's some value in
aligning the two.

\emph{this is really crying out for an example and/or diagrams,
something that shows commits unfolding and let's you illustrate a diff,
sort of like this from the barlett talk}

\includegraphics[width=1\linewidth]{bartlett-commit-history}

Consider version A of a file and a modified version, version B. Assume
that version A was part of a Git commit and version B was part of the
next commit. The set of differences between A and B is called a ``diff''
and Git users contemplate diffs a lot. Diff inspection is how you
re-explain to yourself how version A differs from version B. Diff
inspection is not limited to adjacent commits. You can inspect the diffs
between any two commits.

In fact, Git's notion of version B of your file is as an accumulation of
diffs. At some earlier point, the file was created in the first place.
That version of the file was part of a commit and, therefore, a diff.
Git stores Version A of the file as the initial version, plus all the
intervening diffs in the history that affect the file. And version B
simply includes one more. We'll set these internal details aside now,
but understanding the importance of these deltas will eventually make
Git's operations less baffling.

So, by looking at diffs, it's easy to see how two snapshots differ, but
what about the why?

Every time you make a commit you must also write a short \textbf{commit
message}. Ideally, this conveys the motivation for the change. Remember,
the diff will show the content. When you revisit a project after a break
or need to digest recent changes made by a colleague, looking at the
\textbf{history}, by reading commit messages and skimming through diffs,
is an extremely efficient way to get up to speed.

Every commit needs some sort of nickname, so you can identify it. Git
does this automatically, assigning each commit what is called a SHA, a
seemingly random string of 40 letters and numbers (it is not, in fact,
random but is a SHA-1 checksum hash of the commit). Though you will be
exposed to these, you don't have to handle them directly very often and,
when you do, usually the first 7 characters suffice. You can also
designate certain snapshots as special with a \textbf{tag}, which is a
name of your choosing. In a software project, it is typical to tag a
release with its version, e.g., ``v1.0.3''. For a manuscript or
analytical project, you might tag the version submitted to a journal or
transmitted to external collaborators.

\subsection{Markdown is special on
GitHub}\label{markdown-is-special-on-github}

This may seem unrelated to Git, GitHub, and R, but we need to talk about
\href{https://daringfireball.net/projects/markdown/syntax}{Markdown}.
Markdown is a markup language, like HTML and LaTeX, but designed to be
as lightweight as possible. The goal is still to separate form and
content, but also to prioritize human-readability, even at the cost of
fancy features. Markdown is in wide use on sites like
\href{https://en.support.wordpress.com/markdown/}{WordPress},
\href{https://stackoverflow.com/editing-help}{StackOverflow}, and, yes,
\href{https://help.github.com/categories/writing-on-github/}{GitHub}.
These sites use Markdown because it allows a diverse population of site
users to author decent-looking web content, with hyperlinks and some
formatting. Do not build this up into some heroic, LaTeX-level learning
task, for it is not. If you can write an email, you can write Markdown.

Any file written in Markdown is rendered in an HTML-like way on GitHub.
In particular, formatting and links ``just work''. This is the last
piece we need to seal my claim that merely pushing your project to
GitHub gives it a web presence for zero extra work. If you make even a
modest effort to embed a few explanatory Markdown files in your repo,
you will get an automatically-updated project website for free. In
particular, if a directory has a \texttt{README.md} file, GitHub renders
it like a home page or ``index.html'' when people visit that directory
in the browser. It is very common for a repo to have a top-level
\texttt{README.md}, but each subdirectory can have its own as well.

\subsection{Markdown is special for R
users}\label{markdown-is-special-for-r-users}

Markdown is extra special for R users because of
\href{http://rmarkdown.rstudio.com}{R Markdown}, which is just Markdown
that includes chunks of R code. Again, do not regard R Markdown as
something you must clear your schedule to learn. If you can write email
and a bit of R code, you can write R Markdown. The
\href{https://CRAN.R-project.org/package=rmarkdown}{rmarkdown package}
\citep{rmd-pkg} converts R Markdown (\texttt{.Rmd} files) to Markdown
(\texttt{.md} files), running the code and inserting the results,
including figures, into the document. This is powered by another
package, \href{https://CRAN.R-project.org/package=knitr}{knitr}
\citep{knitr-pkg, knitr-book}, under the hood. This process is made
especially easy in RStudio, but is by no means limited to users of that
application.

\emph{this too is crying out for a visual Rmd -\textgreater{} md
-\textgreater{} rendered thing looking good}

These R-derived Markdown files, if committed and pushed, then enjoy the
usual privileged treatment on GitHub already described above. Once an
\texttt{.Rmd} file has been rendered to \texttt{.md}, anyone viewing it
on GitHub can read the prose, study the R code, \textbf{and view the
results of running that code}, including figures. It is the best of all
worlds, because the code is revealed and, by definition, is the code
that produced the results. And yet a reader can gaze upon the product in
a web browser, without needing to download the code, install all
necessary dependencies, and run it.

The overall effect is that a directory that is a GitHub-synced Git repo
can simultaneously be the code-heavy back end of a project and an
outward-facing front end.

You do not, in fact, even need to work in R Markdown to exploit this. It
\href{http://rmarkdown.rstudio.com/articles_report_from_r_script.html}{works
with plain R scripts} as well. You can use exactly the same machinery to
prepare a rendered version of an R script, i.e.~to go from \texttt{.R}
to \texttt{.md}. Again, RStudio makes this especially easy, but this is
not limited to RStudio. Once the Markdown file is pushed to GitHub, it
is as if the reader has run your code or is able to look over your
shoulder at your R session. This provides an lightweight system for
exposing work-in-progress to collaborators, without slowing down to
create separate reports. Comment lines that begin with
\texttt{\#\textquotesingle{}} are elevated to top-level prose, providing
a way to make the document more welcoming for a reader. Once there are
many prose comments, you might decide to switch from \texttt{.R} to
\texttt{.Rmd}, have proper top-level prose, and move the code down into
chunks.

\emph{figure showing the yaml needed for this? so the use of
\texttt{github\_document} or \texttt{keep\_md\ =\ TRUE}; try to make a
figure that does double or triple duty, i.e.~illustrates this and points
made elsewhere}

Let's zoom back out again and consider R Markdown and the rmarkdown
package more generally. I want to point out that R Markdown can be
rendered to many more formats than Markdown! I have emphasized the
production of Markdown here because it is extremely useful in
GitHub-hosted projects and that seems to be underappreciated. But the
rmarkdown package can convert \texttt{.Rmd} to a wide array of
\href{http://rmarkdown.rstudio.com/lesson-9.html}{output formats},
including HTML, PDF, and Word (\texttt{.docx}). The Markdown must be
created regardless. Sometimes it is all you need, such as on GitHub. In
other contexts, it is just a necessary intermediate and may even be
discarded.

\subsection{Which files to commit}\label{which-files-to-commit}

The files in a project play different roles and arise in different ways.
Let's have a few examples in mind for this discussion:

\begin{itemize}
\tightlist
\item
  R markdown \texttt{.Rmd} --\textgreater{} markdown \texttt{.md}
\item
  R markdown \texttt{.Rmd} --\textgreater{} markdown \texttt{.md}
  --\textgreater{} \texttt{.html} or \texttt{.pdf} or \texttt{.docx}
\item
  R script \texttt{.R} --\textgreater{} results as \texttt{.csv} or
  \texttt{.rds} and figures \texttt{.png}
\item
  LaTeX \texttt{.tex} --\textgreater{} \texttt{.aux}, \texttt{.bbl},
  etc. --\textgreater{} \texttt{.pdf}
\end{itemize}

The files at the far left are clearly source files. In the case of an R
script, this is literally true, but it's morally true for R markdown and
LaTeX files too. These are files that you directly create and edit by
hand.

The files at the far right are clearly derived and are often described
as \textbf{targets}. These files are programmatically generated from
source files (and possibly other inputs). These files are the product
and they have external value, often for communicating ideas and results.

The files in the middle are intermediates. Like targets, they are
programmatically generated, but, unlike targets, no one necessarily
cares about them. However, note that intermediate Markdown \texttt{.md}
is an exception, since it is extremely useful on GitHub -- much more so
than \texttt{.html}, \texttt{.pdf}, or \texttt{.docx} -- and is more
like an additional target.

A critical issue for workflow happiness is figuring out how to manage
the production and storage of source, intermediates, and targets with
respect to Git. You can direct Git to ignore specific files or certain
types of files, such as autosaves created by your editor. This reduces
clutter in your project: Git will not pester you to commit changes to
these files and they will not appear in the associated GitHub
repository. A file that Git does not ignore is said to be
\textbf{tracked}.

The only point on which there is consensus is that source files should
absolutely be tracked. The best treatment of intermediates and targets
with respect to Git is much less clear cut.

Therefore, the main message for intermediates and targets is that you
can pick a policy that works for you and adapt as your needs change.
There is no right answer. I suggest erring on the side of committing
everything at first.

What are the main considerations when deciding whether to track a
derived file or file type?

\emph{I suspect this is a good place to be less wordy? I am so
frustrated with people doing stupid things here, though!}

\emph{Is it immediately useful to someone?} If so, that is a reason to
track it and push it to GitHub. There is a taboo against committing
derived products, inherited from Git's software development roots. The
reasoning is that compiled programs are platform-specific and,
therefore, people are better served by getting current source from Git
and compiling themselves. I think data analytic targets, like figures
and rendered reports, are very different beasts and it's misguided to
reflexively exclude them from version control. They are immediately
useful, especially to consumers of a project (versus the makers). To the
extent that a GitHub repo is meant for dissemination, there is no reason
to burden every consumer with unnecessary friction. Most simply will not
bother to clone the repo, install all the necessary dependencies, and
remake the products. Make them readily available.

\emph{Is it available elsewhere?} If so, then perhaps you don't need to
track it and push it to GitHub. Many people who have a policy of not
tracking derived products also have a system that makes these available
elsewhere, such as on a separate website. There are ways to automate
this via GitHub, but that is beyond the scope of this article and not
recommended for your early days with Git and GitHub. Beware those who
recommend the exclusion of derived products without offering any
practical solution for making them available some other way.

\emph{Is it huge or changing often? Is it a format that is of little use
on GitHub?} These are all good reasons to not track a file with Git.
They can make your repository bloated and slow down pushes and pulls. If
a file is binary, such as a Word document or Excel spreadsheet, Git and
Git clients will not be able to provide human-readable diffs. Neither
will GitHub be able to directly display this file in the browser. Be
aware, however, GitHub-friendliness does not just boil down to ``is it
plain text?''. GitHub has excellent support for non-code files, such as
image formats (PNG, JPG, GIF, and SVG) and PDFs. It even provides visual
diffs, which are extremely useful for understanding changes in figures.
Finally, even though HTML is plain text, it is of little direct use on
GitHub, because, unlike Markdown, it is not rendered.

\emph{Will diffs be useful to you?} Some derived files are simply too
big or miserable to read casually, such as \texttt{.csv} files of
processed results or \texttt{.html} derived from \texttt{.Rmd}. But they
may still be worth tracking with Git, because the diffs are often modest
and quite interesting. I have caught unexpected changes in analytical
results and student-facing webpages this way. When you re-run an
analysis with updated input data or after updating R packages, the diffs
presented by Git help you quickly pinpoint the downstream consequences.

\emph{Will it make collaboration harder?} Prose to be written: Talk
about the scenario Nick brought up re: binary files like PDF being a
common source of merge conflict
\url{https://github.com/dsscollection/git-github-for-stats/issues/6}. I
have a solution! Stop using PDF as your default output format! Make
Markdown the output format during development. Your Git/GitHub problems
just went away. You're welcome.

In summary, I recommend you default to including a file in your Git
repository, unless there's a specific reason not to. But good reasons
absolutely do exist.

\subsection{Collaboration}\label{collaboration}

Collaboration is probably the most compelling reason to manage a project
with Git and GitHub. I have a broad definition of collaboration. It
includes hands-on participation by multiple people as well as an
asymmetric model, in which some people are active makers and others only
read or review.

Consider two different models of collaboration on a document:

\begin{itemize}
\item
  \textbf{Edit, save, attach.} In this workflow, everyone has one (or
  more!) copies of the document and they circulate via email attachment,
  accumulating initials and dates in the filename. Which one is
  ``master''? Does this question even make sense anymore? How do
  different versions of the document relate to each other? If you want
  to see a version combining the best versions of each section, how
  would you reconcile the different copies into one? All of this usually
  gets sorted out by social contract, a fairly manual process, and at
  least one miserable person.
\item
  \textbf{Google Doc.} In this workflow, there is only one copy of the
  document and it lives in the cloud. Anyone can access the most recent
  version on demand. Anyone can edit or comment or propose a change and
  this is immediately available to everyone else. Anyone can see who's
  been editing the document and how and, if disaster strikes, can revert
  to a previous version. A great deal of ambiguity and annoying
  reconciliation work has been designed away.
\end{itemize}

Managing a project via Git/GitHub is much more like the Google Doc
scenario, but also offers some of the attractive features of ``edit,
save, attach''. With Git/GitHub, collaborators can work offline and
there can be independent lines of development. The real power comes from
regular and structured reconciliation of all versions of all the files
in the project via Git/GitHub. It is definitely more complicated than
collaborating on a Google Doc, but also more powerful.

How does collaboration work?

Git is a decentralized version control system, meaning each collaborator
has their own complete copy of the repo and its history. Everyone can
work offline and/or simultaneously, but with regular syncing to GitHub.
GitHub plays the role of another collaborator, but a very special one.
By convention, everyone agrees that GitHub keeps the master copy of the
project. GitHub is the clearinghouse. The joke is that GitHub puts the
``central'' in decentralized version control. You pull regularly from
GitHub, to receive and integrate changes made by your collaborators. You
also push regularly to GitHub, to return the favor, and to maintain its
status as the comprehensive, authoritative version of the project.

\emph{These two points were originally written as comparisons with
Google Docs, but now there are intervening paragraphs. Is it too
disconnected? Is it worth weaving in?}

\emph{Manage multiple files}. A Git repository is inherently multi-file
and therefore well suited to projects comprised of many files, evolving
in a coordinated fashion. Examples include a data analysis, a course
website, a blog, an R package, or a book. If there is any way to
proactively check or enforce their joint functionality, this is
something you could verify manually prior to a commit or at certain
milestones. In the case of a website, you might choose to rebuild the
site prior to a commit. In the case of an R package,
\texttt{R\ CMD\ check} is one of the easiest things to automate on
GitHub.

\emph{Diffs and time travel}. Google Docs are fantastic for simple
collaborative work, when you don't need detailed access to the history.
But the version control offered by Google Drive is very limited compared
to Git. You can't compare versions at arbitrary points in time,
temporarily checkout previous versions, or maintain two lines of
development.

\emph{May need a segue \ldots{} depends on the fate of the two
paragraphs above.}

Merge conflicts are the most frustrating thing about using Git and
GitHub. You can avoid them if you only work alone, on one computer, but
I've also said that collaboration is the best reason to use GitHub! So
this problem must be confronted.

What is a merge conflict? Here is a typical first encounter: your
collaborator makes a change to a file, commits it locally, and pushes to
GitHub. Meanwhile you also make a different change to the same file and
also commit locally. When you try to push your commit to GitHub, it will
fail because there are commits on GitHub that you do not have. You must
pull from GitHub. The good news is that quite often, this will ``just
work'', i.e.~the GitHub version and your version will merge cleanly. Git
is quite clever at reconciliation and changes to different files or even
distinct parts of the same file will merge. If this pull and merge goes
smoothly, you'll able to push your changes and the cycle goes on.

But sometimes it's not clear how to reconcile your changes with the new
ones from GitHub and you get a \emph{merge conflict}. You must inspect
the locations of conflict, which Git marks for you. You will pick one
version or the other -- or create a hybrid -- at each location of
conflict and mark it as resolved. Once you've resolved all conflicts,
you will be able to push a version integrating your recent changes to
GitHub.

The best way to deal with merge conflicts is to avoid them altogether.
This is another reason for all parties to commit, pull, and push often.
Small changes, integrated frequently, in non-binary files, are the
easiest for Git to automatically merge for you. The difficulty of
merging (by Git or by you) is proportional to the evolutionary distance
between two lines of work. The presence of frequently-changing binary
files also increases the burden. So make lots of small commits, sync
regularly with GitHub, and only track binary files with good reason.

\emph{once I understand Nick's point about binary files causing merge
conflicts
\url{https://github.com/dsscollection/git-github-for-stats/issues/6} I
may need to reword the above}

\emph{something I would like to write about, but have no space for here:
different models of collaboration, i.e.~everyone working on one repo \&
one branch vs.~making pull requests from branches in the main repo or
from forks}

\emph{inspiration for possible figures on collaboration/merging and the
role of a remote, from the barlett talk}

\includegraphics[width=1\linewidth]{bartlett-merge-commit}

\includegraphics[width=1\linewidth]{bartlett-pull}

\subsection{GitHub as course management
system}\label{github-as-course-management-system}

\href{http://stat545.com}{STAT 545} is a data wrangling and analysis
course at the University of British Columbia. I was the instructor in
charge for several years, which coincided with my own adoption of
Git/GitHub. GitHub is used to manage the development of course material,
to serve the course website, to create a discussion forum, and to host
all student-submitted work.

Given that students must submit their work and provide peer review of
others' work via GitHub, the use of hosted version control is an
explicit, though modest, part of the course. The website
\href{http://happygitwithr.com}{Happy Git and GitHub for the useR} holds
our battle-tested instructions for setup and early usage. The students
achieve basic competence quite quickly and find it gratifying to see
their formatted, figure-rich R Markdown reports up on the internet.
Since it's easy to expose their work within the class, we conduct peer
review, which helps expertise to spread quickly through the group.

\subsubsection{Use a GitHub
Organization}\label{use-a-github-organization}

\href{https://help.github.com/articles/differences-between-user-and-organization-accounts/}{GitHub
Organizations} are ``shared accounts where groups of people can
collaborate across many projects at once''. This is the most appropriate
structure for stewarding course resources, since I can grant TAs and
students different levels of access to various repositories. Access can
be controlled at the individual user level or, more conveniently, for
entire \href{https://help.github.com/articles/setting-up-teams/}{Teams}.
The TA Team shares write access with me on a private repository for
internal matters. I provide each student with their own private
repository for coursework and grant other members of the Students Team
read access, in order to facilitate peer review. There is a public
repository that underpins the course website (see below). We have one
other public repository that exists solely so the
\href{https://github.com/STAT545-UBC/Discussion/issues}{Issues} can be
used as a discussion forum.

GitHub actively encourages the use of its platform in teaching. As an
instructor you can request a
\href{https://help.github.com/articles/discounted-organization-accounts/}{free
Organization account} that provides features normally available only on
paid plans, such as private repositories. In fact, GitHub provides
tooling for specific teaching workflows via
\href{https://classroom.github.com/}{GitHub Classroom}, although I do
not use it. That is not an intentional knock on their tools. I started
teaching with GitHub several years before this existed and developed a
different way of using the platform. I also find the
\href{https://education.github.com}{GitHub Education} resources to be
geared more towards computer science than data science.

\subsubsection{GitHub Pages for course
website}\label{github-pages-for-course-website}

All course content is provided on the \href{http://stat545.com}{STAT 545
website}. Each page is generated from an R Markdown document that is
rendered to HTML locally using the rmarkdown package, retaining the
intermediate Markdown. These pages are a mix of prose and rendered R
code, reflecting the live coding done in class. All of these files and
their history can be explored in the
\href{https://github.com/STAT545-UBC/STAT545-UBC.github.io}{source
repository}. The TA team has permission to write to this repo, meaning
they can (and do!) help me maintain the website. I rejoice that I am no
longer the webmaster. We also get typo corrections and other input from
the world at large, since this is entirely public.

If I were starting from scratch today, I would continue to use R
Markdown, RStudio, and GitHub Pages (see below), but would upgrade to a
more modern, automated approach to rendering the pages. I now recommend
\href{http://rmarkdown.rstudio.com/rmarkdown_websites.html}{R Markdown
websites}, \href{https://bookdown.org}{bookdown}, or
\href{https://bookdown.org/yihui/blogdown/}{blogdown} to manage the
process of creating a static website from a large and inter-related set
of \texttt{.Rmd} files.

GitHub offers several ways to host a website directly from a repository,
collectively known as
\href{https://help.github.com/categories/github-pages-basics/}{GitHub
Pages}. The STAT 545 website is a very simple
\href{https://help.github.com/articles/user-organization-and-project-pages/}{Organization
Page} that uses a
\href{https://help.github.com/articles/custom-domain-redirects-for-github-pages-sites/}{custom
domain}, \texttt{stat545.com}, instead of the default
\texttt{orgname.github.io}.

This system for managing course content is a great example of
integrating the doing of work and the sharing of it. We analyze data
live in class, using R, based on the scripts on the website. I re-render
the associated \texttt{.R} or \texttt{.Rmd}, commit the changed files,
push, and see it reflected right away on \url{http://stat545.com}. There
is no separation between having an idea, implementing it, and posting on
the website.

\subsubsection{Student-specific private
repos}\label{student-specific-private-repos}

Early in the course I elicit GitHub usernames for registered students,
via a \href{https://shiny.rstudio.com}{Shiny app}, and invite them to
join the course Organization. I then create one private repository per
student, in the STAT 545 Organization. The targeted student has write
access and the other students have read access. This is somewhat
controversial, due to the possibility of cheating, but I have seen more
pros than cons for this setup, in the STAT 545 context. In other
settings, I have also used one repo per student \emph{per homework
assignment}, which allows you to keep the repos completely private until
homework submission, then increase their visibility during marking and
peer review. Some courses will work better with one model or the other.

Each student does their work in this repo, submitting a major assignment
approximately once a week. The first assignment is simply to claim the
repository and create a README, which proves they have all the relevant
software setup and they can write a little Markdown. Each week we tackle
some new data analysis or wrangling task, with increasing latitude for
independence. Homework is implemented in R Markdown documents, rendered
to Markdown, and pushed to GitHub. Students submit their work by opening
an issue in their repo, naming the assignment in the title, providing
the SHA of the associated final commit, and linking to the main
\texttt{.md} file. We leave feedback as comments in the issue thread or,
occasionally, propose changes to code via ``pull requests''. Two peers
are selected at random to review each assignment, a process that we also
implement via GitHub Issues.

At the end of term, the student (and their instructor!) can visit the
repo to find an organized, navigable sequence of \textasciitilde{}10
assignments. Each student leaves with self-written documentation of
everything they've done, ready to consult in future projects. The last
assignments require writing an R package or Shiny app, which they
generally do in public repositories under their own accounts. They
finish STAT 545 with several months of Git/GitHub experience and the
start of a data science portfolio.

\subsection{GitHub as web presence}\label{github-as-web-presence}

Simply having a project on GitHub gives it a web presence! Non-users of
Git/GitHub can visit the project in the browser and interact with it
like a webpage. They can grab a snapshot of all the files as a ZIP
archive by simply clicking a button. People with GitHub accounts have
even more options. They can clone or fork the repository to get their
own copy, which also makes it easy for them to stay current on future
changes.

As described above, GitHub Pages offer various ways to serve proper
websites, even quite sophisticated ones, directly out of a GitHub repo.
But before you even worry about that, certain practices can make a
GitHub repository much more browsable. For many projects, this is more
than sufficient for granting people access to your work.

\textbf{Be savvy about file formats.} Keep files in the plainest,
web-friendliest form that is compatible with your main goals. As
explained above, Markdown is the ideal format for prose, because it is
just plain text with some markup, but will be displayed like HTML on
GitHub. Files named \texttt{README.md} are extra special, acting as the
index or landing page for their host directory.
\href{https://help.github.com/articles/rendering-csv-and-tsv-data/}{CSV
and TSV} files also get special treatment, including an attractive grid
layout and search. GitHub has excellent support for displaying and
diffing common image formats.

\textbf{Use conventional file extensions.} GitHub is very code-focused
and will apply proper syntax high-lighting for almost any language you
can think of, if you use one of the standard file extensions. This also
has advantages for people searching GitHub and trying to filter by
language.

\textbf{Use internal links.} \texttt{README.md} is a great place to
explain how your project fits together. Any Markdown file can include
relative links to other files in the repo. Embedded images are also
displayed. For figures produced by R code, these links are part of what
rmarkdown takes care of for you, but there's no reason you can't do the
same yourself for any Markdown file.

*link to \url{http://happygitwithr.com/repo-browsability.html*}

\subsection{More resources}\label{more-resources}

I've tried to convey the main points about the use of Git and GitHub in
statistical and data analytical settings, but I've had to leave many
things out. There are more advanced topics that will come up once your
use of Git becomes more sophisticated and there are topics that are only
relevant to certain types of reader.

I targeted \href{https://github.com}{GitHub} -- not
\href{https://bitbucket.org}{Bitbucket} or
\href{https://about.gitlab.com}{GitLab} -- for the sake of specificity.
However, all the big-picture principles and even some mechanics will
carry over to these alternative hosting platforms. I am advocating for
the use of hosted version control as a general concept, with GitHub
being the best and most common provider today. I note that many
companies and even universities are starting to make GitHub Enterprise
or GitLab available internally. For example, we host our own instance of
GitHub Enterprise at UBC to support our Master of Data Science program

Don't fret too much about public versus private repositories at this
point. All the major hosting providers offer private repositories with
flexible control over who can read or write to the repo. There are many
ways to get private repositories for low or no cost, especially for
academics. If you outgrow this initial arrangement, you can throw some
combination of technical savvy and money at the problem. You can either
pay for a higher level of service, self-host one of these platforms, or
advocate for organization-wide solutions.

Branches and pull requests are an extremely powerful feature of
Git/GitHub and should be your first foray beyond the basics of commit,
push, and pull. A branch is an independent line of development within a
repo, with the notion it will eventually be merged back into the main
branch, a.k.a. ``master''. In a course website, you might work in a
branch to update a series of lessons, while leaving the current version
intact in the meantime. A pull request is a specific and formal GitHub
process for merging one branch into another. You can make a pull request
between two branches in the same repo or between two copies of a repo.
This is the mechanism for making and accepting contributions to open
source software projects on GitHub, including many popular R packages
\emph{link to a search?}.

\subsection{Conclusion}\label{conclusion}

\emph{NEEDS ONE \ldots{} BUT SHORT \ldots{} SOME FODDER}

Transparency about process and product is increasingly important in
science. The SOMETHING for reproducibility is well accepted. A more
underappreciated benefit is democratization of our field, as this
affords a much broader audience a clear view of how scientists and
programmers work.

make statistical thought and implementation available

\subsection{Random things lying
around}\label{random-things-lying-around}

Consider links to GitHub repos that exemplify certain points and are
very unlikely to disappear any time soon.n

possible github explainer
\url{https://www.wired.com/2015/03/github-conquered-google-microsoft-everyone-else/}

\url{http://stackoverflow.com/questions/2712421/r-and-version-control-for-the-solo-data-analyst}

\url{https://www.onshape.com/cad-blog/how-google-solved-the-version-control-problem}

\url{https://github.com/blog/2289-publishing-with-github-pages-now-as-easy-as-1-2-3}

The active R package development community on GitHub.

\begin{itemize}
\tightlist
\item
  If you care deeply about someone else's project, such as an R package
  you use heavily, you can track its development on GitHub. You can
  watch the repository to get notified of major activity. You can fork
  it to keep your own copy. You can modify your fork to add features or
  fix bugs and send them back to the owner as a proposed change.
\end{itemize}

The read-only mirrors of R source and all of CRAN. Coupled with GitHub
search features, you can answer alot of your own questions this way.

\bibliographystyle{agsm}
\bibliography{git-github-for-stats.bib}

\end{document}
